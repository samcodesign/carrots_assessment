\documentclass[11pt]{article}
\usepackage[margin=1in]{geometry}
\usepackage{hyperref}
\usepackage{array}
\usepackage{booktabs}
\usepackage{amsmath}
\usepackage{setspace}
\setstretch{1.1}

\title{Product Analyst Assessment -- Paywall A/B Test}
\author{Candidate: Samy Benmahrez}
\date{}

\begin{document}
\maketitle

\section*{1. Context and Data}
This analysis evaluates an A/B test of two paywall variants in a mobile app: \emph{Standard} and \emph{Plans}. The goal is to understand impact on (1) conversion from active users to paying users, (2) revenue net of taxes and platform commissions, and (3) retention via subscription renewals.

Two CSV datasets were provided: \texttt{user\_behaviour.csv} (event-level telemetry with event\_date, event\_name, test\_variant, paywall\_type, product\_identifier, and user identifiers) and \texttt{revenuecat.csv} (subscription-level data with product\_identifier, price\_in\_usd, tax\_percentage, commission\_percentage, renewal\_number, refund flags, and timestamps). All transformations and aggregations were done in a Python notebook (pandas) and exported to KPI summary CSVs, then used to build a Looker Studio dashboard.

\section*{2. Methodology and KPI Definitions}
\textbf{User identity:} \texttt{revenue\_cat\_id} is the main user key across both datasets.\\
\textbf{Variant assignment:} each \texttt{revenue\_cat\_id} is assigned a single test\_variant (Standard or Plans) based on events in \texttt{user\_behaviour}.

\subsection*{2.1 Conversion Rate}
Conversion rate (per variant):
\[
\text{Conversion}_{v} = \frac{\#\{\text{users with at least one }payment\_success\}}{\#\{\text{users with any event in variant }v\}}
\]
Users are distinct by \texttt{revenue\_cat\_id}; all exposed users are the denominator and \texttt{payment\_success} is the conversion signal.

\subsection*{2.2 Purchases by Paywall}
Purchases by paywall type = number of distinct users with \texttt{payment\_success}, grouped by (\texttt{test\_variant}, \texttt{paywall\_type}). This shows which paywall surfaces drive revenue per variant.

\subsection*{2.3 Revenue, ARPU, and ARPPU}
For each non-refunded subscription row in RevenueCat:
\[
\text{net\_revenue} = \text{price\_in\_usd} \times \bigl(1 - \text{tax\_percentage} - \text{commission\_percentage}\bigr)
\]
Using this:
\begin{align*}
\text{Total net revenue}_{v} &= \sum \text{net\_revenue}_i \quad \text{for subscriptions in variant } v \\
\text{ARPU}_{v} &= \frac{\text{Total net revenue}_{v}}{\text{users}_{v}} \\
\text{ARPPU}_{v} &= \frac{\text{Total net revenue}_{v}}{\text{paying users}_{v}}
\end{align*}
Refunded subscriptions (\texttt{refunded\_at} not null) are excluded.

\subsection*{2.4 Most Purchased Package}
Product identifiers include suffixes \_a/\_b/\_c. Normalise to a base SKU:
\[
\text{product\_base} = \text{product\_identifier without trailing } a/b/c
\]
Count distinct subscribers per (\texttt{test\_variant}, \texttt{product\_base}) to find the dominant package.

\subsection*{2.5 Retention via renewal\_number}
For each user and variant, compute \(\text{max\_renewal} = \max(\text{renewal\_number})\) across non-refunded subscriptions. Then per variant:
\[
\text{Retention }k+ = \frac{\#\{\text{paying users with } \text{max\_renewal} \ge k\}}{\#\{\text{paying users}\}}, \quad k \in \{1,2,3\}
\]
These approximate renewal-based retention across billing cycles. In production these definitions would be validated with Product and Finance stakeholders before finalising.

\section*{3. Results -- Standard vs Plans}
\subsection*{3.1 Top-line KPIs}
\begin{center}
\begin{tabular}{>{\raggedright\arraybackslash}p{1.7cm}cccccc}
\toprule
Variant & Users & Paying users & Conversion & Net revenue (\$) & ARPU (\$) & ARPPU (\$) \\
\midrule
Plans & 1{,}637 & 260 & 15.88\% & 3{,}339.67 & 2.04 & 12.84 \\
Standard & 1{,}544 & 405 & 26.23\% & 3{,}804.54 & 2.46 & 9.39 \\
\bottomrule
\end{tabular}
\end{center}

\textbf{Interpretation.} Standard converts significantly more users (~26.2\% vs 15.9\%), generates more net revenue (~\$3.80k vs \$3.34k), and higher ARPU (2.46 vs 2.04). Plans monetises each payer more (ARPPU 12.84 vs 9.39) but on fewer payers. Overall, Standard is the stronger default: more users convert, more renew, and total revenue is higher even though Plans extracts more per payer.

\section*{4. Paywalls and Packages}
\subsection*{4.1 Purchases by Paywall Type}
Using \texttt{kpi\_purchases\_by\_paywall}, onboarding is the main driver of purchases in both variants; feature and laggard paywalls are secondary. Plans is lower across all paywall types, consistent with its lower conversion. \textbf{Implication:} the onboarding paywall is the highest-leverage surface (copy, pricing, layout, risk reduction).

\subsection*{4.2 Most Purchased Package}
From \texttt{kpi\_packages} (normalised \texttt{product\_base}), the dominant package is \texttt{gala\_1wt\_1w\_gold} in both variants (394 subscribers in Standard, 219 in Plans). Other packages (\texttt{gala\_1w\_creator}, \texttt{gala\_1y\_creator}, \texttt{gala\_1y\_gold}) have much smaller volume. \textbf{Implication:} revenue is concentrated on one core weekly Gold offer; optimising this SKU is the most leveraged path.

\section*{5. Retention}
From \texttt{kpi\_retention} (based on \texttt{max\_renewal}):
\begin{itemize}
    \item \textbf{Standard:} paying users 402; retention 1+ \(\approx 22.6\%\); 2+ \(\approx 11.7\%\); 3+ \(\approx 6.0\%\).
    \item \textbf{Plans:} paying users 251; retention 1+ \(\approx 20.3\%\); 2+ \(\approx 7.6\%\); 3+ \(\approx 2.4\%\).
\end{itemize}
Standard converts more and retains a higher share through successive renewals; Plans drops off faster after the first renewal, limiting LTV despite higher ARPPU.

\section*{6. Synthesis and Recommendations}
\begin{itemize}
    \item \textbf{Set Standard as the default} variant: it wins on conversion, total net revenue, ARPU, and renewal-based retention.
    \item \textbf{Use Plans selectively} for high-intent segments (heavy users, high-WTP geos/channels) to leverage its higher ARPPU.
    \item \textbf{Focus optimisation on onboarding} paywalls: value messaging, trial/guarantee, social proof, and showing value before asking.
    \item \textbf{Double down on \texttt{gala\_1wt\_1w\_gold}}: experiment with price sensitivity, framing, limited-time offers, or bundles anchored on the weekly Gold plan.
    \item \textbf{Future analyses:} segment by country/acquisition channel; cohort-based retention by start month and variant; evaluate non-revenue impacts if available.
\end{itemize}

\section*{7. Looker Studio Dashboard}
KPIs and breakdowns are exposed in an interactive dashboard (filters for variant and date; pages for overview, paywalls \& packages, retention). Dashboard link: \href{https://lookerstudio.google.com/reporting/4702f26e-4c79-4142-85d8-5dc097929d57}{https://lookerstudio.google.com/reporting/4702f26e-4c79-4142-85d8-5dc097929d57}.

\end{document}
